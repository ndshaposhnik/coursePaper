%%%%%%%%%%%%%%%%%%%%%%%%%%%%%%%%%%%%%%%%%
% Beamer Presentation
% LaTeX Template
% Version 1.0 (10/11/12)
%
% This template has been downloaded from:
% http://www.LaTeXTemplates.com
%
% License:
% CC BY-NC-SA 3.0 (http://creativecommons.org/licenses/by-nc-sa/3.0/)
%
%%%%%%%%%%%%%%%%%%%%%%%%%%%%%%%%%%%%%%%%%

%----------------------------------------------------------------------------------------
%	PACKAGES AND THEMES
%----------------------------------------------------------------------------------------

\documentclass{beamer}

\mode<presentation> {

%\usetheme{default}
%\usetheme{AnnArbor}
%\usetheme{Antibes}
%\usetheme{Bergen}
%\usetheme{Berkeley}
%\usetheme{Berlin}
%\usetheme{Boadilla}
%\usetheme{CambridgeUS}
%\usetheme{Copenhagen}
\usetheme{Darmstadt}
%\usetheme{Dresden}
%\usetheme{Frankfurt}
%\usetheme{Goettingen}
%\usetheme{Hannover}
%\usetheme{Ilmenau}
%\usetheme{JuanLesPins}
%\usetheme{Luebeck}
%\usetheme{Madrid}
%\usetheme{Malmoe}
%\usetheme{Marburg}
%\usetheme{Montpellier}
%\usetheme{PaloAlto}
%\usetheme{Pittsburgh}
%\usetheme{Rochester}
%\usetheme{Singapore}
%\usetheme{Szeged}
%\usetheme{Warsaw}

% As well as themes, the Beamer class has a number of color themes
% for any slide theme. Uncomment each of these in turn to see how it
% changes the colors of your current slide theme.

%\usecolortheme{albatross}
%\usecolortheme{beaver}
%\usecolortheme{beetle}
%\usecolortheme{crane}
%\usecolortheme{dolphin}
%\usecolortheme{dove}
%\usecolortheme{fly}
%\usecolortheme{lily}
%\usecolortheme{orchid}
%\usecolortheme{rose}
%\usecolortheme{seagull}
%\usecolortheme{seahorse}
%\usecolortheme{whale}
%\usecolortheme{wolverine}

%\setbeamertemplate{footline} % To remove the footer line in all slides uncomment this line
%\setbeamertemplate{footline}[page number] % To replace the footer line in all slides with a simple slide count uncomment this line

%\setbeamertemplate{navigation symbols}{} % To remove the navigation symbols from the bottom of all slides uncomment this line
}

\usepackage{graphicx} % Allows including images
\usepackage{booktabs} % Allows the use of \toprule, \midrule and \bottomrule in tables

\usepackage[T2A]{fontenc}
\usepackage[utf8]{inputenc}
\usepackage[russian, english]{babel}
\usepackage{amsmath}
\usepackage{amssymb}
\usepackage{amsthm}
\usepackage{amsfonts}

\usepackage{tikz}

\everymath{\color{blue}}

\newtheorem*{Th}{Теорема}
\newtheorem*{Def}{Определение}
\newtheorem*{Prop}{Утверждение}
\newtheorem*{Rem}{Замечание}
\newtheorem*{Exmpl}{Пример}

\newcommand{\cred}[1]{\textcolor{red}{#1}}
\newcommand{\cblue}[1]{\textcolor{blue}{#1}}
\newcommand{\cwhite}[1]{\textcolor{white}{#1}}
\newcommand{\cblack}[1]{\textcolor{black}{#1}}


\title[Short title]{Степени тропических матриц и графы} % The short title appears at the bottom of every slide, the full title is only on the title page

\author{Никита Шапошник, МФТИ\\научный руководитель -- А.Э. Гутерман} 

\date{29 ноября 2021} % Date, can be changed to a custom date

\begin{document}

\begin{frame}
\titlepage
\end{frame}

\section{Введение}

\begin{frame}
\frametitle{Тропическое полукольцо}
$\mathbb{R}_{\max} = \mathbb{R} \cup \{ -\infty\}$ с операциями сложения $\oplus$ и умножения $\odot$: \begin{align*}
            a \oplus b &= \max(a, b)\\
            a \odot b &= a + b
        \end{align*}
$\mathbb{R}_{\max}$ также называется max-plus алгеброй.\\

$\mathbb{R}_{min} = \mathbb{R} \cup \{ \infty\}$ с операциями сложения $\oplus$ и умножения $\odot$: \begin{align*}
            a \oplus b &= min(a, b)\\
            a \odot b &= a + b
        \end{align*}
\end{frame}

%------------------------------------------------

\begin{frame}
\frametitle{Свойства тропического полукольца} 
Для любых $a, b, c \in \mathbb{R}_{\max}$ верно:
\begin{itemize}
\item Сложение и умножение ассоциативны.
\item Сложение и умножение коммутативны.
\item Дистрибутивность: $a \odot (b \oplus c) = a \odot b \oplus a \odot c$.
\item $-\infty$ --- нулевой элемент: $a \oplus -\infty = a$.
\item $0$ --- единичный элемент: $a \odot 0= a$.
\item Результат умножения на тропический ноль --- тропический ноль: $a \odot -\infty = -\infty$.
\item Несуществование обратного по сложению: если $a \neq -\infty$, то $a \oplus b \ge a > -\infty$.
\end{itemize}
\end{frame}

%------------------------------------------------


\begin{frame}
\frametitle{Определения теории графов} 
\begin{enumerate}
	\item \cred{Ориентированный граф} $\mathcal{G} = {\mathcal{G}(V,E)}$. Петли разрешены, кратные рёбра -- нет.
	\item $\mathcal{W}^t(i \rightarrow j)$ --- множество всех путей из вершины $i$ в вершину $j$ длины $t$;\\
	\cred{Длина пути} --- это количетсво ребер в нём; \\
	$\mathcal{W}(i \rightarrow j)$ --- множество всех путей из вершины $i$ в вершину $j$. Вершины и ребра в обоих случаях могут повторяться.
	\item Граф $\mathcal{G}(V, E)$ со введенной функцией $p : E \rightarrow \mathbb{R}$ называется \cred{взвешенным} графом.\\
	\cred{Весом пути} $W = e_1e_2\dots e_k$ называется число 

	\begin{equation*}
		p(W) = \bigodot_{i = 1}^k p(e_i).
	\end{equation*}
\end{enumerate}

\end{frame}

%------------------------------------------------

\begin{frame}
\frametitle{Связь матриц и графов}
$A \in M_d(\mathbb{R}_{max}) \longleftrightarrow \mathcal{G} = {\mathcal{G}(V,E)}$, если
\begin{enumerate}
	\item $|V| = d$;
	\item $a_{ij} \neq -\infty \Leftrightarrow (i, j) \in E$ и $a_{ij} = p((i, j))$;
	\item $a_{ij} = -\infty \Leftrightarrow (i, j) \notin E$.
\end{enumerate}
Такая матрица $A$ называется \cred{матрицей смежности} графа $\mathcal{G}$. Граф, построенный по матрице $A$, обозначим через $\mathcal{G}(A)$.

\end{frame}

%------------------------------------------------

\begin{frame}
\frametitle{Степени тропических матриц и графы}
\begin{Prop}
Рассмотрим $A \in M_{d \times d}(\mathbb{R}_{\max})$, $i, j \in V(\mathcal{G}(A))$, $t \in \mathbb{N} \cup \{0\}$. Тогда:\begin{equation*}
    a_{ij}^t = \bigoplus \{p(W): W \in \mathcal{W}^t(i \rightarrow j)\}
\end{equation*}
\end{Prop}
\begin{proof}[\textbf{Доказательство}]
База: $t = 0$: $A^0 = I = diag(0, 0, ..., 0) = \begin{pmatrix}
0 & -\infty & ... & -\infty \\
-\infty & 0 &... & -\infty \\
...     & ... & ... & ... \\
-\infty & -\infty & ...  & 0
\end{pmatrix}$.\\
Переход:
\begin{equation*}
    a_{ij}^{t + 1} = \bigoplus_{k = 1}^{d} a_{ik}^t \odot a_{kj} = \max_k (a_{ik}^t + a_{kj})
\end{equation*}
\end{proof}

\end{frame}

%------------------------------------------------
\begin{frame}
\frametitle{Максимальный средний вес цикла}
\begin{Def}
\cred{Максимальный средний вес цикла} в $\mathcal{G}$:
\begin{equation*}
    \begin{split}
        \lambda(A) = \bigoplus_{k = 1}^d \bigoplus_{i_1, \dots, i_k} (a_{{i_1}{i_2}}\odot \dots \odot a_{{i_k}{i_1}})^{\odot{1/k}} =\\
        =\max_{k = 1}^d \max_{i_1, \dots, i_k} \frac{(a_{{i_1}{i_2}} + \dots + a_{{i_k}{i_1}})}{k}
    \end{split}
\end{equation*}
\end{Def}
\end{frame}

%------------------------------------------------

\begin{frame}
\frametitle{Звезда Клини}
\begin{Def}
Для $A \in M_{d \times d}(\mathbb{R}_{\max})$ с $\lambda(A) \le 0$ определим звезду Клини:
\begin{equation*}
	A^* = \bigoplus_{i = 0}^{\infty} A^i
\end{equation*}
\end{Def}
\begin{Prop}
\begin{equation*}
	A^* = \bigoplus_{i = 0}^{d - 1} A^i
\end{equation*}
\begin{equation*}
	(A^*)_{ij} = p(\mathcal{W}(i \rightarrow j))
\end{equation*}
\end{Prop}
\end{frame}

%------------------------------------------------
\begin{frame}
\frametitle{Критические циклы и критический подграф}
\begin{Def}
	Ориентированный цикл называется \cred{критическим}, если у него максимальный средний вес.
\end{Def}

\begin{Def}
	Объединение всех критических циклов называется \cred{критическим подграфом}.
\end{Def}

 
\end{frame}

%------------------------------------------------

\begin{frame}
\frametitle{Примитивные матрицы} 
\begin{Def}
\begin{itemize}
	\item Матрица $A \in M_{n\times n}(\mathbb{R}), A \ge 0$ называется \cred{примитивной}, если $\exists k \in \mathbb{N}: A^k > 0$.
	\item Матрица $A \in M_{n\times n}(\mathbb{R}_{\max})$ называется 				\cred{примитивной}, если $\exists k \in \mathbb{N}: A^k$ не содержит $-\infty$.
	\item Наименьшее такое $k$ называется \cred{экспонентой} $A$ и обозначается через $exp(A)$.
\end{itemize}
\end{Def}

\begin{Th}[Виландта]
	Экспонента примитивной матрицы порядка $n$ не превосходит $Wi(n) = n^2 -2n + 2$.
\end{Th}
\end{frame}

%------------------------------------------------

\begin{frame}
\frametitle{Примитивность на языке графов} 
Матрица $A$ примитивна тогда и только тогда, когда существует такое $k \in \mathbb{N}$, что в $\mathcal{G}(A)$ для любых вершин $u$ и $v$ есть путь из $u$ в $v$ длины $k$.
\end{frame}

%------------------------------------------------

\begin{frame}
\frametitle{Сильная связность и неразложимость}
	\begin{Def}
		Граф $\mathcal{G}$ зовётся \cred{сильно связным}, если для любых $u, v \in V(\mathcal{G})$ есть путь из $u$ в $v$.
	\end{Def}
	
	\begin{Def}
Матрица $A \in M_d(\mathbb{R_{\max}})$ (или соответствующий ей граф) \cred{неразложима}, если граф $\mathcal{G}(A)$ сильно связен, иначе \cred{разложима}.\\
Матрица $A \in M_d(\mathbb{R_{\max}})$ (или соответствующий ей граф) \cred{полностью разложима}, если в графе $\mathcal{G}(A)$ нет рёбер между различными компонентами сильной связности.\\
	\end{Def}
\end{frame}

%------------------------------------------------

\begin{frame}
\frametitle{Индекс цикличности}
\begin{Def}
	\cred{Индекс цикличности} $\sigma_{\mathcal{G}}$ графа $\mathcal{G}$ равен:
	\begin{itemize}
		\item 1, если в $\mathcal{G}$ есть только одна вершина (с петлей или без). 
		\item НОД всех длин ориентированных циклов в $\mathcal{G}$, если $\mathcal{G}$ сильно связен, и $|V(\mathcal{G})| \ge 2$.
		\item НОК цикличностей всех максимальных его сильно связных подграфов,если $\mathcal{G}$ не сильно связен.
	\end{itemize}
\end{Def}
\begin{Th}
Неразложимая матрица примитивна тогда и только тогда, когда ее индекс цикличности равен $1$.
\end{Th}
\end{frame}

%------------------------------------------------
%\begin{frame}
%\frametitle{Скрамблинг индекс}
%\begin{Def}
%	\cred{Скрамблинг индекс} $k(\mathcal{G})$ ориентированного графа $\mathcal{G}$ --- это наименьшее $k \in \mathbb{N} :$ $\forall u, v \in V(\mathcal{G}) \hookrightarrow \exists w \in V(\mathcal{G})$ такая, что есть путь длины $k$ из $u$ в $w$ и из $v$ в $w$. Если не существует таких $k$, то $k(\mathcal{G}) = 0$.
%\end{Def}
%\begin{Rem}
%Если $\mathcal{G}$ --- примитивный ориентированный граф, то
%\begin{equation*}
%	0 < k(\mathcal{G}) \le exp(\mathcal{G}).
%\end{equation*}	
%\end{Rem}

%\end{frame}

%------------------------------------------------

\section{CSR-декомпозиция}
\begin{frame}
\frametitle{Матрицы CSR}
Пусть $A \in M_d(\mathbb{R}_{\max})$, $\mathcal{D}$ --- подграф $\mathcal{G}^c(A)$ без тривиальных компонент сильной связности. 
\\Введем обозначение: $M = ((\lambda(A)^-\odot A)^\sigma)^*$, где \\
\begin{itemize}
\item $\lambda(A)$ --- максимальный средний вес цикла в $\mathcal{G}(A)$.\\
\item $\lambda(A)^- = -\lambda(A)$ --- обратный к $\lambda(A)$ по умножению элемент;\\
\item $\sigma$ --- цикличность критического подграфа $\mathcal{G}^c(A)$.\\
\item $A^* = \bigoplus_{i = 0}^{\infty} A^i$ --- звезда Клини матрицы $A$.
\end{itemize}

Определим матрицы $C, S, R \in M_r(\mathbb{R}_{\max})$:
\begin{align*}
    c_{ij} &= \begin{cases}
        m_{ij}\text{, если } j \in V(\mathcal{D})\\
        -\infty \text{, иначе,}
    \end{cases}
    &
    r_{ij} = \begin{cases}
        m_{ij}\text{, если } i \in V(\mathcal{D})\\
        -\infty \text{, иначе,}
    \end{cases}
    \\
    s_{ij} &= \begin{cases}
        \lambda(A)^- \odot a_{ij}\text{, если } (i, j) \in E(\mathcal{D})\\
        -\infty \text{, иначе.}
    \end{cases}
\end{align*}
\end{frame}

%------------------------------------------------

\begin{frame}
\frametitle{Пример}

Рассмотрим следующий граф $\mathcal{G}(A)$:

\begin{center}
\begin{tikzpicture}[scale=0.12]
\tikzstyle{every node}+=[inner sep=0pt]
\draw [black] (39,-7.2) circle (3);
\draw (39,-7.2) node {$1$};
\draw [black] (61.4,-17.8) circle (3);
\draw (61.4,-17.8) node {$2$};
\draw [black] (56.8,-46.8) circle (3);
\draw (56.8,-46.8) node {$3$};
\draw [black] (22.1,-46.8) circle (3);
\draw (22.1,-46.8) node {$4$};
\draw [black] (16.7,-19.2) circle (3);
\draw (16.7,-19.2) node {$5$};
\draw [black] (37.677,-4.52) arc (234:-54:2.25);
\draw (39,0.05) node [above] {$0$};
\fill [black] (40.32,-4.52) -- (41.2,-4.17) -- (40.39,-3.58);
\draw [black] (63.48,-15.655) arc (163.61718:-124.38282:2.25);
\draw (68.52,-14.77) node [right] {$0$};
\fill [black] (64.37,-18.15) -- (64.99,-18.85) -- (65.28,-17.89);
\draw [black] (59.48,-45.477) arc (144:-144:2.25);
\draw (64.05,-46.8) node [right] {$-1$};
\fill [black] (59.48,-48.12) -- (59.83,-49) -- (60.42,-48.19);
\draw [black] (53.8,-46.8) -- (25.1,-46.8);
\fill [black] (25.1,-46.8) -- (25.9,-47.3) -- (25.9,-46.3);
\draw (39.45,-46.3) node [above] {$-3$};
\draw [black] (23.28,-44.04) -- (37.82,-9.96);
\fill [black] (37.82,-9.96) -- (37.05,-10.5) -- (37.97,-10.89);
\draw (31.28,-27.95) node [right] {$-3$};
\draw [black] (19.577,-48.402) arc (-29.85515:-317.85515:2.25);
\draw (14.74,-47.62) node [left] {$-2$};
\fill [black] (19.29,-45.77) -- (18.85,-44.94) -- (18.35,-45.81);
\draw [black] (13.716,-19.355) arc (300.69698:12.69698:2.25);
\draw (9.77,-15.57) node [left] {$-3$};
\fill [black] (14.76,-16.93) -- (14.78,-15.98) -- (13.92,-16.49);
\draw [black] (19.34,-17.78) -- (36.36,-8.62);
\fill [black] (36.36,-8.62) -- (35.42,-8.56) -- (35.89,-9.44);
\draw [black] (36.36,-8.62) -- (19.34,-17.78);
\fill [black] (19.34,-17.78) -- (20.28,-17.84) -- (19.81,-16.96);
\draw (26.51,-12.7) node [above] {$-7$};
\draw [black] (41.71,-8.48) -- (58.69,-16.52);
\fill [black] (58.69,-16.52) -- (58.18,-15.72) -- (57.75,-16.63);
\draw [black] (58.69,-16.52) -- (41.71,-8.48);
\fill [black] (41.71,-8.48) -- (42.22,-9.28) -- (42.65,-8.37);
\draw (51.19,-11.99) node [above] {$0$};
\draw [black] (57.27,-43.84) -- (60.93,-20.76);
\fill [black] (60.93,-20.76) -- (60.31,-21.47) -- (61.3,-21.63);
\draw [black] (60.93,-20.76) -- (57.27,-43.84);
\fill [black] (57.27,-43.84) -- (57.89,-43.13) -- (56.9,-42.97);
\draw (58.4,-32.09) node [left] {$-1$};
\draw [black] (58.49,-18.529) arc (-76.8156:-99.59658:98.38);
\fill [black] (58.49,-18.53) -- (57.6,-18.22) -- (57.83,-19.2);
\draw [black] (58.489,-18.525) arc (-76.88805:-99.52412:98.997);
\fill [black] (19.65,-19.74) -- (20.36,-20.37) -- (20.52,-19.38);
\draw (39.16,-21.6) node [below] {$-7$};
\draw [black] (55.57,-44.06) -- (40.23,-9.94);
\fill [black] (40.23,-9.94) -- (40.1,-10.87) -- (41.01,-10.46);
\draw [black] (40.23,-9.94) -- (55.57,-44.06);
\fill [black] (55.57,-44.06) -- (55.7,-43.13) -- (54.79,-43.54);
\draw (47.18,-28) node [left] {$-1$};
\draw [black] (53.981,-45.773) arc (-110.95184:-138.12582:91.248);
\fill [black] (53.98,-45.77) -- (53.41,-45.02) -- (53.06,-45.95);
\draw [black] (53.982,-45.771) arc (-111.00042:-138.07725:91.567);
\fill [black] (18.67,-21.46) -- (18.83,-22.39) -- (19.57,-21.73);
\draw (33.53,-36.21) node [below] {$-7$};
\draw [black] (17.28,-22.14) -- (21.52,-43.86);
\fill [black] (21.52,-43.86) -- (21.86,-42.97) -- (20.88,-43.17);
\draw [black] (21.52,-43.86) -- (17.28,-22.14);
\fill [black] (17.28,-22.14) -- (16.94,-23.03) -- (17.92,-22.83);
\draw (20.13,-32.7) node [right] {$-7$};
\end{tikzpicture}
\end{center}

\end{frame}

%------------------------------------------------

\begin{frame}
\frametitle{Пример}
Ему соответствует матрица $A \in M_5(\mathbb{R}_{\max})$:
\begin{equation*}
A = \begin{pmatrix}
0 & 0 & -1 & -\infty & -7 \\
0 & 0 & -1 & -\infty & -7 \\
-1 & -1 & -1 & -3 & -7 \\
-3 & -\infty & -\infty & -2 & -7 \\
-7 & -7 & -7 & -7 & -3 \\
\end{pmatrix}
\end{equation*}
$\lambda(A) = 0$.
\end{frame}

%------------------------------------------------

\begin{frame}
\frametitle{Пример}
Критический подграф: его цикличность равна: $\sigma = 1$.\\
Возьмем $\mathcal{D} = \mathcal{G}^c(A)$.
\begin{center}
\begin{tikzpicture}[scale=0.12]
\tikzstyle{every node}+=[inner sep=0pt]
\draw [black] (39.3,-18.4) circle (3);
\draw (39.3,-18.4) node {$1$};
\draw [black] (60.5,-30.2) circle (3);
\draw (60.5,-30.2) node {$2$};
\draw [black] (54.6,-55) circle (3);
\draw (54.6,-55) node {$3$};
\draw [black] (24.4,-55) circle (3);
\draw (24.4,-55) node {$4$};
\draw [black] (18.6,-30.8) circle (3);
\draw (18.6,-30.8) node {$5$};
\draw [black] (41.92,-19.86) -- (57.88,-28.74);
\fill [black] (57.88,-28.74) -- (57.42,-27.92) -- (56.94,-28.79);
\draw [black] (57.88,-28.74) -- (41.92,-19.86);
\fill [black] (41.92,-19.86) -- (42.38,-20.68) -- (42.86,-19.81);
\draw (50.9,-23.8) node [above] {$0$};
\draw [black] (37.977,-15.72) arc (234:-54:2.25);
\draw (39.3,-11.15) node [above] {$0$};
\fill [black] (40.62,-15.72) -- (41.5,-15.37) -- (40.69,-14.78);
\draw [black] (61.557,-27.405) arc (187.02507:-100.97493:2.25);
\draw (65.97,-24.13) node [right] {$0$};
\fill [black] (63.36,-29.34) -- (64.22,-29.74) -- (64.09,-28.74);
\end{tikzpicture}
\end{center}

\end{frame}

%------------------------------------------------

\begin{frame}
\frametitle{Пример}
\begin{equation*}
M = ((\lambda(A)^-\odot A^\sigma)^* = A^* = \begin{pmatrix}
   0 &   0 &  -1 &  -4 &  -7 \\
   0 &   0 &  -1 &  -4 &  -7 \\
  -1 &  -1 &   0 &  -3 &  -7 \\
  -3 &  -3 &  -4 &   0 &  -7 \\
  -7 &  -7 &  -7 &  -7 &   0
\end{pmatrix}
\end{equation*}
\end{frame}

%------------------------------------------------

\begin{frame}
\frametitle{Пример}
\begin{align*}
   	C &= \begin{pmatrix}
    		 0 &   0 & -\infty & -\infty & -\infty \\
  		 0 &   0 & -\infty & -\infty & -\infty \\
  		-1 &  -1 & -\infty & -\infty & -\infty \\
  		-3 &  -3 & -\infty & -\infty & -\infty \\
  		-7 &  -7 & -\infty & -\infty & -\infty
    \end{pmatrix}
    \\
    R &= \begin{pmatrix}
   0 &   0 &  -1 &  -4 &  -7 \\
   0 &   0 &  -1 &  -4 &  -7 \\
  -\infty &  -\infty & -\infty & -\infty & -\infty \\
  -\infty &  -\infty & -\infty & -\infty & -\infty \\
  -\infty &  -\infty & -\infty & -\infty & -\infty
\end{pmatrix}
\end{align*}
\end{frame}

%------------------------------------------------

\begin{frame}
\frametitle{Пример}
\begin{align*}
   S = \begin{pmatrix}
   0 &   0 & -\infty & -\infty & -\infty \\
   0 &   0 & -\infty & -\infty & -\infty \\
  -\infty &  -\infty & -\infty & -\infty & -\infty \\
  -\infty &  -\infty & -\infty & -\infty & -\infty \\
  -\infty & -\infty & -\infty & -\infty & -\infty
\end{pmatrix}
\end{align*}
\begin{Rem}
	В дальнейшем матрицы $C, S, R$, определённые через матрицу $A$, будем обозначать через $CSR[A]$.
\end{Rem}
\end{frame}

%------------------------------------------------


\begin{frame}
\frametitle{Граница $\cwhite{T}$}
\begin{Th} [Sergeev, 2009]
Пусть $A \in M_d(\mathbb{R}_{\max})$ неразложима и CSR-матрицы определены через некоторый подграф $\mathcal{D}$ графа $\mathcal{G}^c(A)$. Тогда $\exists T(A) \hookrightarrow \forall t \ge T(A):$
\begin{equation*}
    A^t = \lambda(A)^{\odot t} \odot CS^tR[A].
\end{equation*}
\end{Th}
\begin{Rem}
	Если $\lambda(A) = 0$, то $\forall t \ge T(A):$
	\begin{equation*}
    A^t = CS^tR[A].
\end{equation*}
\end{Rem}
\end{frame}

%------------------------------------------------
%\subsection{Границы $\cwhite{T_1}$ и $\cwhite{T_2}$}
\begin{frame}
\frametitle{Вспомогательная матрица $\cwhite{B}$}
Введем новую матрицу $B \in M_d(\mathbb{R}_{\max})$:

\[
\cblue{b_{ij}} = \begin{cases}
		-\infty \cblack{\text{, если $i$ или $j$ лежат в } \mathcal{D},} \\
   		a_{ij}, \cblack{\text{ иначе.}}
\end{cases}
\]
\begin{equation*}
B = \begin{pmatrix}
-\infty & -\infty & -\infty & -\infty & -\infty \\
-\infty & -\infty & -\infty & -\infty & -\infty \\
-\infty & -\infty & -1 & -3 & -7 \\
-\infty & -\infty & -\infty & -2 & -7 \\
-\infty & -\infty & -7 & -7 & -3 \\
\end{pmatrix}
\end{equation*}
Эта матрица нужна нам для определения следующих границ.
\end{frame}

%------------------------------------------------

\begin{frame}
\frametitle{Границы $\cwhite{T_1}$ и $\cwhite{T_2}$}
\begin{Th}[Merlet, Nowak, Sergeev, 2014]
Пусть $A \in M_d(\mathbb{R}_{\max})$ неразложима. Тогда \\$\exists T_1(A, B) \hookrightarrow \forall t \ge T_1(A, B):$
\begin{equation*}
    A^t = (\lambda(A)^{\odot t} \odot CS^tR[A]) \oplus B^t.
\end{equation*}
$\exists T_2(A, B) \hookrightarrow \forall t \ge T_2(A, B):$
\begin{equation*}
    \lambda(A)^{\odot t} \odot CS^tR[A] \ge B^t.
\end{equation*}

\end{Th}
Если $\lambda(A) = 0$, то равенства из определения $T_1$ и $T_2$ записываются в виде:
\begin{gather*}
A^t = CS^tR[A] \oplus B^t \\
	  CS^tR[A] \ge B^t
\end{gather*}
\end{frame}

%------------------------------------------------

\begin{frame}
\frametitle{Как относятся между собой разные границы}
\begin{Prop}
$T(A) \le \max(T_1(A, B), T_2(A, B))$.
\end{Prop}
\begin{proof}[\textbf{Доказательство}]
Возьмем $t \ge \max(T_1(A, B), T_2(A, B))$. Тогда:
\begin{gather*}
A^t = (\lambda(A)^{\odot t} \odot CS^tR[A]) \oplus B^t \\
\lambda(A)^{\odot t} \odot CS^tR[A] \ge B^t
\end{gather*}
и,значит,
\begin{equation*}
A^t = \lambda(A)^{\odot t} \odot CS^tR[A]
\end{equation*}
\end{proof}
\begin{Exmpl}
Для матрицы $A$ граница $T(A) = 5$, $T_1(A, B) = 2$, а $T_2(A, B) = 5$
\end{Exmpl}
\end{frame}

%------------------------------------------------

\begin{frame}
\frametitle{Как выбрать подграф $\cwhite{\mathcal{D}}$?}
$\mathcal{D} = \mathcal{G}^c(A)$ --- способ Нахтигалля.
\begin{block}{Обозначения}
Обозначим через $B_N$ матрицу $B$, выбранную способом Нахтигалля.\\
Будем писать $T_{1, N}(A)$ вместо $T_1(A, B_N)$ и $T_{2, N}(A)$ вместо $T_2(A, B_N)$.
\end{block}
\end{frame}

%------------------------------------------------

\begin{frame}
\frametitle{Инвариантность относительно умножения на скаляр}
\begin{Prop}[Kennedy-Cochran-Patrick, Merlet, Nowak, Sergeev]
Если $A' = \mu \odot A$, где $\mu \in \mathbb{R}$, то 

\begin{itemize}
	\item $\lambda(A') = \mu \odot \lambda(A)$
	\item $B_N[A'] = B_N[A]$
	\item $CSR[A'] = CSR[A]$
\end{itemize}

Значит, $T_1(A, B), T_2(A, B)$ инвариантны относительно умножении матрицы на скаляр, что позволяет нам без разграничения общности говорить, что $\lambda(A) = 0$.

\end{Prop}
\end{frame}

%------------------------------------------------
%\begin{frame}
%\frametitle{Новые обозначения}
%Введем несколько новых обозначений:
%\begin{enumerate}
%    \item Через $\mathcal{W}^{t, l}(i \rightarrow j)$ обозначим множество путей от вершины $i$ к вершине $j$, имеющих длину $t$ по модулю $l$;
%    \item Через $\mathcal{W}(i \xrightarrow{\mathcal{G}} j)$ обозначим множество путей от вершины $i$ к вершине $j$, проходящих хотя бы через одну вершину из $\mathcal{G}$. Аналогично определяются $\mathcal{W}^t(i \xrightarrow{\mathcal{G}} j)$, $\mathcal{W}^{t, l}(i \xrightarrow{\mathcal{G}} j)$.
%    \item Для множества $\mathcal{W}$ через $p(\mathcal{W})$ обозначим максимальный вес пути из множества $\mathcal{W}$.
%\end{enumerate}
%\end{frame}

%------------------------------------------------

\begin{frame}
\frametitle{Смысл матриц $\cwhite{CSR}$}
\begin{Prop}[Kennedy-Cochran-Patrick, Merlet, Nowak, Sergeev]
Если $\lambda(A) = 0$, то верно следующее тождество:
\begin{equation}
    (CS^tR[A])_{ij} = p(\mathcal{W}^{t, \sigma}(i \xrightarrow{\mathcal{G}^c(A)} j)),
\end{equation}
где $\sigma$ обозначает цикличность $\mathcal{G}^c(A)$, $p(\mathcal{W})$ --- максимальный вес пути из множества $\mathcal{W}$,
\end{Prop}
\end{frame}

%------------------------------------------------

\begin{frame}
\frametitle{Некоторые оценки для $\cwhite{T_{1, N}(A)}$}
\begin{Th}[Kennedy-Cochran-Patrick, Merlet, Nowak, Sergeev]
Для любой $A \in M_n(\mathbb{R}_{\max})$ имеем:
\begin{enumerate} 
    \item $T_{1, N}(A) \le n^2 - 2n + 2$;
    \item $T_{1, N}(A) \le \hat{g}(n - 2) + n$, где $\hat{g} = \hat{g}(\mathcal{G}^c(A))$ --- обхват критического подграфа, то есть наименьшая длина цикла.
\end{enumerate}
\end{Th}
\end{frame}

%------------------------------------------------

\begin{frame}
\frametitle{Границы T для цикла}
\begin{center}
\begin{tikzpicture}[scale=0.11]
\tikzstyle{every node}+=[inner sep=0pt]
\draw [black] (10.6,-14.6) circle (3);
\draw (10.6,-14.6) node {$9$};
\draw [black] (55.8,-15.5) circle (3);
\draw (55.8,-15.5) node {$2$};
\draw [black] (42.9,-4.5) circle (3);
\draw (42.9,-4.5) node {$1$};
\draw [black] (60.4,-30.3) circle (3);
\draw (60.4,-30.3) node {$3$};
\draw [black] (54.6,-46.4) circle (3);
\draw (54.6,-46.4) node {$4$};
\draw [black] (42.1,-56.5) circle (3);
\draw (42.1,-56.5) node {$5$};
\draw [black] (23.2,-56.5) circle (3);
\draw (23.2,-56.5) node {$6$};
\draw [black] (11.2,-47.6) circle (3);
\draw (11.2,-47.6) node {$7$};
\draw [black] (4.1,-32.2) circle (3);
\draw (4.1,-32.2) node {$8$};
\draw [black] (24.7,-4.5) circle (3);
\draw (24.7,-4.5) node {$10$};
\draw [black] (45.18,-6.45) -- (53.52,-13.55);
\fill [black] (53.52,-13.55) -- (53.23,-12.65) -- (52.58,-13.41);
\draw [black] (56.69,-18.36) -- (59.51,-27.44);
\fill [black] (59.51,-27.44) -- (59.75,-26.52) -- (58.79,-26.82);
\draw [black] (59.38,-33.12) -- (55.62,-43.58);
\fill [black] (55.62,-43.58) -- (56.36,-42.99) -- (55.42,-42.66);
\draw [black] (52.27,-48.29) -- (44.43,-54.61);
\fill [black] (44.43,-54.61) -- (45.37,-54.5) -- (44.74,-53.72);
\draw [black] (39.1,-56.5) -- (26.2,-56.5);
\fill [black] (26.2,-56.5) -- (27,-57) -- (27,-56);
\draw [black] (20.79,-54.71) -- (13.61,-49.39);
\fill [black] (13.61,-49.39) -- (13.95,-50.27) -- (14.55,-49.46);
\draw [black] (9.94,-44.88) -- (5.36,-34.92);
\fill [black] (5.36,-34.92) -- (5.24,-35.86) -- (6.15,-35.44);
\draw [black] (5.14,-29.39) -- (9.56,-17.41);
\fill [black] (9.56,-17.41) -- (8.81,-17.99) -- (9.75,-18.34);
\draw [black] (13.04,-12.85) -- (22.26,-6.25);
\fill [black] (22.26,-6.25) -- (21.32,-6.31) -- (21.9,-7.12);
\draw [black] (27.7,-4.5) -- (39.9,-4.5);
\fill [black] (39.9,-4.5) -- (39.1,-4) -- (39.1,-5);
\end{tikzpicture}
\end{center}
Будем считать, что $\lambda(A) = 0$.
\end{frame}

%------------------------------------------------

\begin{frame}
\frametitle{Границы T для цикла}
\begin{itemize}
\item $\mathcal{G}^c(A) = \mathcal{G}(A)$,
\item $\sigma = n$,
\item
$M = (A^n)^* = E^* = E = \begin{pmatrix}
0 & -\infty & ... & -\infty \\
-\infty & 0 & ... & -\infty \\
... & ... & ... & ... \\
-\infty & -\infty & ... & 0
\end{pmatrix}$,
\item $C = R = E$, $S = A$, $B = -\infty$.
\end{itemize}
Значит, $CS^tR[A] = A^t$ для любого неотрицательного $t$.\\
Следовательно, $T = T_1 = T_2 = 0$.
\end{frame}

%------------------------------------------------

\begin{frame}
\frametitle{Границы T для двустороннего цикла}
\begin{center}
\begin{tikzpicture}[scale=0.1]
\tikzstyle{every node}+=[inner sep=0pt]
\draw [black] (10.6,-14.6) circle (3);
\draw (10.6,-14.6) node {$9$};
\draw [black] (55.8,-15.5) circle (3);
\draw (55.8,-15.5) node {$2$};
\draw [black] (42.9,-4.5) circle (3);
\draw (42.9,-4.5) node {$1$};
\draw [black] (60.4,-30.3) circle (3);
\draw (60.4,-30.3) node {$3$};
\draw [black] (54.6,-46.4) circle (3);
\draw (54.6,-46.4) node {$4$};
\draw [black] (42.1,-56.5) circle (3);
\draw (42.1,-56.5) node {$5$};
\draw [black] (23.2,-56.5) circle (3);
\draw (23.2,-56.5) node {$6$};
\draw [black] (11.2,-47.6) circle (3);
\draw (11.2,-47.6) node {$7$};
\draw [black] (4.1,-32.2) circle (3);
\draw (4.1,-32.2) node {$8$};
\draw [black] (24.7,-4.5) circle (3);
\draw (24.7,-4.5) node {$10$};
\draw [black] (45.18,-6.45) -- (53.52,-13.55);
\fill [black] (53.52,-13.55) -- (53.23,-12.65) -- (52.58,-13.41);
\draw [black] (56.69,-18.36) -- (59.51,-27.44);
\fill [black] (59.51,-27.44) -- (59.75,-26.52) -- (58.79,-26.82);
\draw [black] (59.38,-33.12) -- (55.62,-43.58);
\fill [black] (55.62,-43.58) -- (56.36,-42.99) -- (55.42,-42.66);
\draw [black] (52.27,-48.29) -- (44.43,-54.61);
\fill [black] (44.43,-54.61) -- (45.37,-54.5) -- (44.74,-53.72);
\draw [black] (39.1,-56.5) -- (26.2,-56.5);
\fill [black] (26.2,-56.5) -- (27,-57) -- (27,-56);
\draw [black] (20.79,-54.71) -- (13.61,-49.39);
\fill [black] (13.61,-49.39) -- (13.95,-50.27) -- (14.55,-49.46);
\draw [black] (9.94,-44.88) -- (5.36,-34.92);
\fill [black] (5.36,-34.92) -- (5.24,-35.86) -- (6.15,-35.44);
\draw [black] (5.14,-29.39) -- (9.56,-17.41);
\fill [black] (9.56,-17.41) -- (8.81,-17.99) -- (9.75,-18.34);
\draw [black] (13.04,-12.85) -- (22.26,-6.25);
\fill [black] (22.26,-6.25) -- (21.32,-6.31) -- (21.9,-7.12);
\draw [black] (27.7,-4.5) -- (39.9,-4.5);
\fill [black] (39.9,-4.5) -- (39.1,-4) -- (39.1,-5);
\draw [black] (39.9,-4.5) -- (27.7,-4.5);
\fill [black] (27.7,-4.5) -- (28.5,-5) -- (28.5,-4);
\draw [black] (53.52,-13.55) -- (45.18,-6.45);
\fill [black] (45.18,-6.45) -- (45.47,-7.35) -- (46.12,-6.59);
\draw [black] (59.51,-27.44) -- (56.69,-18.36);
\fill [black] (56.69,-18.36) -- (56.45,-19.28) -- (57.41,-18.98);
\draw [black] (55.62,-43.58) -- (59.38,-33.12);
\fill [black] (59.38,-33.12) -- (58.64,-33.71) -- (59.58,-34.04);
\draw [black] (44.43,-54.61) -- (52.27,-48.29);
\fill [black] (52.27,-48.29) -- (51.33,-48.4) -- (51.96,-49.18);
\draw [black] (26.2,-56.5) -- (39.1,-56.5);
\fill [black] (39.1,-56.5) -- (38.3,-56) -- (38.3,-57);
\draw [black] (13.61,-49.39) -- (20.79,-54.71);
\fill [black] (20.79,-54.71) -- (20.45,-53.83) -- (19.85,-54.64);
\draw [black] (5.36,-34.92) -- (9.94,-44.88);
\fill [black] (9.94,-44.88) -- (10.06,-43.94) -- (9.15,-44.36);
\draw [black] (9.56,-17.41) -- (5.14,-29.39);
\fill [black] (5.14,-29.39) -- (5.89,-28.81) -- (4.95,-28.46);
\draw [black] (22.26,-6.25) -- (13.04,-12.85);
\fill [black] (13.04,-12.85) -- (13.98,-12.79) -- (13.4,-11.98);
\end{tikzpicture}
\end{center}
\begin{itemize}
\item Оба цикла длины $2$ в этом графе имеют средний вес $0$.
\item $\mathcal{G}^c(A) = \mathcal{G}(A)$.
\end{itemize}
\end{frame}

%------------------------------------------------

\begin{frame}
\frametitle{Границы T для двустороннего цикла, n нечетно}
\begin{itemize}
\item $\sigma = 1$
\item $C = R = M = A^*$
\item $S = A$
\item $B = -\infty$
\item В $CS^tR[A] = A^*A^tA^*$ нет $-\infty$. Значит, $CS^tR[A] = A^*$.
\item Следовательно,$CS^tR[A] = A^t$ верно тогда и только тогда, когда $A^t = A^*$.
\item $T = exp(A)$
\item $exp(A) = n - 1$
\end{itemize}
Значит, $T = T_1 = n - 1$, $T_2 = 0$.
\end{frame}

%------------------------------------------------

\begin{frame}
\frametitle{Границы T для двустороннего цикла, n четно}
\begin{itemize}
\item $\sigma = 2$
\item $C = R = M = (A^2)^*$
\item $S = A^2$
\item $B = -\infty$
\item $A^t = CS^tR = \begin{cases}
(A^2)^* \text{, если } t \text{ четно,}\\
A \odot (A^2)^*\text{, если } t \text{ нечетно.}\\
\end{cases}$ при $t \ge T(A)$.
\end{itemize}
$T(A) = T_1(A) = \frac{n}{2}$, $T_2(A) = 0$.
\end{frame}

%------------------------------------------------

\begin{frame}
\frametitle{Примеры границ T}
\begin{Th}
Если граф $\mathcal{G}(A)$ 
\begin{itemize}
\item сильно связен
\item совпадает со своим критическим подграфом
\item $\lambda(A) = 0$
\item его цикличность $\sigma = 1$
\item для произвольных двух вершин верно, что все пути между ними имеют одинаковый вес
\end{itemize}
то $T(A) = T_{1, N}(A) = exp(A)$, а $T_{2,N}(A) = 0$.
\end{Th}
\end{frame}

%------------------------------------------------


\begin{frame}
\begin{center}
\Huge{Спасибо за внимание!}
\end{center}
\end{frame}

\end{document} 
